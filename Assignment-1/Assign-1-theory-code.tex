\documentclass[11pt]{article}
\usepackage[utf8]{inputenc}
\title{High School Assignment}
\author{K Vivek Kumar}
\date{28 March 2022}
\usepackage{amsmath}
\usepackage{graphicx}
\begin{document}
\maketitle
\section{2018-ICSE-10th board-Problem : 8(b)}
Given, the mean of the following distribution is, m = 24.\\
We know that,
\begin{equation} \label{eqn}
mean(m) = \frac{\sum f_ix_i}{\sum f_i}
\end{equation}\\
\text{ As per the question,}
\begin{table}[h!]
\caption{Given data}
\center
\begin{tabular}{c c c c}
\hline
Intervals & Frequency (f_i) & Mid-Value (x_i) & f_ix_i\\
\hline
0-10 & 7 & 5 & 35\\
\hline
10-20 & a & 15 & 15a\\
\hline
20-30 & 8 & 25 & 200\\
\hline
30-40 & 10 & 35 & 350\\
\hline
40-50 & 5 & 45 & 225\\
\hline
 & \sum f_i = 30 + a &  & \sum f_ix_i = 810 + 15a\\
\end{tabular}
\end{table}\\
Therefore, from equation \ref{eqn}, the value of mean(m = 24) can be written as,\\
\begin{equation*}
24 = \frac{810 + 15a}{30 + a}\\
\end{equation*}
\begin{equation*}
24(30 + a) = 810 + 15a
\end{equation*}
\begin{equation*}
720 + 24a = 810 + 15a
\end{equation*}
\begin{equation*}
9a = 90
\end{equation*}
\begin{equation*}
a = 10
\end{equation*}
Therefore, the required value(a) is 10.
\end{document}

