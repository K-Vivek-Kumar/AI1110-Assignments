\documentclass[12pt, a4paper, twocolumn]{article}
\usepackage[utf8]{inputenc}
\title{High School Assignment}
\author{K Vivek Kumar}
\date{28 March 2022}
\usepackage{amsmath}
\usepackage{graphicx}
\setlength{\columnwidth}{1cm}
\begin{document}
	\maketitle
	\section{2018-ICSE-10th board-Problem : 8(b)}
	\textbf{\underline{Problem:} }If the mean of the following distribution is 24, find the value of 'a'.
	\begin{table}[htb]
		\centering
		\resizebox{\columnwidth}{!}{
			\begin{tabular}{|c|c|c|c|c|c|}
				\hline
				Marks & 0-10 & 10-20 & 20-30 & 30-40 & 40-50\\
				\hline
				Number of students & 7 & a & 8 & 10 & 5\\
				\hline
			\end{tabular}
		}
	\end{table}\\
	\textbf{\underline{Solution:} }Given, the mean of the following distribution is, m = 24.\\
	We know that,
	\begin{equation} \label{eqn}
		mean(m) = \frac{\sum f_ix_i}{\sum f_i}
	\end{equation}
\text{This can also be written as}\\
\begin{equation} \label{eqn}
	mean(m) = \frac{\vec{f}^T\vec{x}}{\vec{1}^T\vec{f}}
\end{equation}
	\text{ As per the question,}
	\begin{table}[htb]
		\centering
		\resizebox{\columnwidth}{!}{
			\begin{tabular}{|c|c|c|}
				\hline
				Intervals & Frequency (f_i) & Mid-Value (x_i)\\
				\hline
				0-10 & 7 & 5 \\
				\hline
				10-20 & a & 15 \\
				\hline
				20-30 & 8 & 25\\
				\hline
				30-40 & 10 & 35\\
				\hline
				40-50 & 5 & 45\\
				\hline
			\end{tabular}\\
		}
	\end{table}\\
Therefore, from the above table we can deduce the following vectors,\\
	\begin{equation*}
		\vec{f} = \begin{bmatrix} 7\\a\\8\\10\\5\\
		\end{bmatrix}; 
	\vec{x} = \begin{bmatrix} 	5\\15\\25\\35\\45\\
	\end{bmatrix}
	\end{equation*}
To find the value of 'a', we can simplify the equation (2),
\begin{equation*}
		mean(m) = \frac{\vec{f}^T\vec{x}\text{(without 'a')} + \vec{f}^T\vec{x}\text{(with 'a')}}{\vec{1}^T\vec{f}\text{(without 'a')} + \vec{1}^T\vec{f}\text{(with 'a')}}
\end{equation*}
Taking the dot product,\\
\begin{equation*}
	\vec{f}^T\vec{x}\text{(without 'a')} = \begin{bmatrix} 7\\8\\10\\5\\
		\end{bmatrix}.\begin{bmatrix} 	5\\25\\35\\45\\
		\end{bmatrix} = [35 + 200 + 350 + 225].
\end{equation*}
\begin{equation*}
	\vec{f}^T\vec{x}\text{(with 'a')} = \begin{bmatrix} a\\
	\end{bmatrix}.\begin{bmatrix} 	15\\
	\end{bmatrix} = [15a].
\end{equation*}
\begin{equation*}
	\vec{1}^T\vec{x}\text{(without 'a')} = \begin{bmatrix} 1\\1\\1\\1\\
	\end{bmatrix}.\begin{bmatrix} 	7\\8\\10\\5\\
	\end{bmatrix} = [7 + 8 + 10 + 5].
\end{equation*}
\begin{equation*}
	\vec{1}^T\vec{x}\text{(with 'a')} = \begin{bmatrix} 1\\
	\end{bmatrix}.\begin{bmatrix} 	a\\
	\end{bmatrix} = [a].
\end{equation*}
On substituting the following above values in the equation, we get
\begin{equation*}
	mean(m) = \frac{[35 + 200 + 350 + 225] + [15a]}{[7 + 8 + 10 + 5] + [a]}
\end{equation*}
\begin{equation*}
	mean(m) = \frac{810 + 15a}{30 + a}
\end{equation*}
	\begin{equation*}
		24(30 + a) = 810 + 15a
	\end{equation*}
	\begin{equation*}
		720 + 24a = 810 + 15a
	\end{equation*}
	\begin{equation*}
		9a = 90
	\end{equation*}
	\begin{equation*}
		a = 10
	\end{equation*}
	Therefore, the required value(a) is 10.
\end{document}
