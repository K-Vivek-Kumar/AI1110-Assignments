\let\negmedspace\undefined
\let\negthickspace\undefined
%\RequirePackage{amsmath}
\documentclass[journal,12pt,twocolumn]{IEEEtran}
%
% \usepackage{setspace}
 \usepackage{gensymb}
%\doublespacing
 \usepackage{polynom}
%\singlespacing
%\usepackage{silence}
%Disable all warnings issued by latex starting with "You have..."
%\usepackage{graphicx}
\usepackage{amssymb}
%\usepackage{relsize}
\usepackage[cmex10]{amsmath}
%\usepackage{amsthm}
%\interdisplaylinepenalty=2500
%\savesymbol{iint}
%\usepackage{txfonts}
%\restoresymbol{TXF}{iint}
%\usepackage{wasysym}
\usepackage{amsthm}
%\usepackage{pifont}
%\usepackage{iithtlc}
% \usepackage{mathrsfs}
% \usepackage{txfonts}
 \usepackage{stfloats}
% \usepackage{steinmetz}
 \usepackage{bm}
% \usepackage{cite}
% \usepackage{cases}
% \usepackage{subfig}
%\usepackage{xtab}
\usepackage{longtable}
%\usepackage{multirow}
%\usepackage{algorithm}
%\usepackage{algpseudocode}
\usepackage{enumitem}
 \usepackage{mathtools}
 \usepackage{tikz}
% \usepackage{circuitikz}
% \usepackage{verbatim}
%\usepackage{tfrupee}
\usepackage[breaklinks=true]{hyperref}
%\usepackage{stmaryrd}
%\usepackage{tkz-euclide} % loads  TikZ and tkz-base
%\usetkzobj{all}
\usepackage{listings}
    \usepackage{color}                                            %%
    \usepackage{array}                                            %%
    \usepackage{longtable}                                        %%
    \usepackage{calc}                                             %%
    \usepackage{multirow}                                         %%
    \usepackage{hhline}                                           %%
    \usepackage{ifthen}   
    \usepackage[english]{babel}
    \usepackage{graphicx}                                        %%
  %optionally (for landscape tables embedded in another document): %%
    \usepackage{lscape}
% \usepackage{multicol}
% \usepackage{chngcntr}
%\usepackage{enumerate}

%\usepackage{wasysym}
%\newcounter{MYtempeqncnt}
\DeclareMathOperator*{\Res}{Res}
\DeclareMathOperator*{\equals}{=}
%\renewcommand{\baselinestretch}{2}

%\renewcommand\thesection{\arabic{section}}
%\renewcommand\thesubsection{\thesection.\arabic{subsection}}
%\renewcommand\thesubsubsection{\thesubsection.\arabic{subsubsection}}

%\renewcommand\thesectiondis{\arabic{section}}
%\renewcommand\thesubsectiondis{\thesectiondis.\arabic{subsection}}
%\renewcommand\thesubsubsectiondis{\thesubsectiondis.\arabic{subsubsection}}

% correct bad hyphenation here
\hyphenation{op-tical net-works semi-conduc-tor}
\def\inputGnumericTable{}                                 %%

\lstset{
%language=C,
frame=single, 
breaklines=true,
columns=fullflexible
}
%\lstset{
%language=tex,
%frame=single, 
%breaklines=true
%}
\title{Assignment 3}
\author{K Vivek Kumar - CS21BTECH11026}

\begin{document}
\date{April 15,2022}
\maketitle
%
\newtheorem{theorem}{Theorem}[section]
\newtheorem{problem}{Problem}
\newtheorem{proposition}{Proposition}[section]
\newtheorem{lemma}{Lemma}[section]
\newtheorem{corollary}[theorem]{Corollary}
\newtheorem{example}{Example}[section]
\newtheorem{definition}[problem]{Definition}
%\newtheorem{thm}{Theorem}[section] 
%\newtheorem{defn}[thm]{Definition}
%\newtheorem{algorithm}{Algorithm}[section]
%\newtheorem{cor}{Corollary}
\newcommand{\BEQA}{\begin{eqnarray}}
\newcommand{\EEQA}{\end{eqnarray}}
\newcommand{\define}{\stackrel{\triangle}{=}}
\newcommand*\circled[1]{\tikz[baseline=(char.base)]{
    \node[shape=circle,draw,inner sep=2pt] (char) {#1};}}
\bibliographystyle{IEEEtran}
%\bibliographystyle{ieeetr}
\providecommand{\mbf}{\mathbf}
\providecommand{\pr}[1]{\ensuremath{\Pr\left(#1\right)}}
\providecommand{\qfunc}[1]{\ensuremath{Q\left(#1\right)}}
\providecommand{\sbrak}[1]{\ensuremath{{}\left[#1\right]}}
\providecommand{\lsbrak}[1]{\ensuremath{{}\left[#1\right.}}
\providecommand{\rsbrak}[1]{\ensuremath{{}\left.#1\right]}}
\providecommand{\brak}[1]{\ensuremath{\left(#1\right)}}
\providecommand{\lbrak}[1]{\ensuremath{\left(#1\right.}}
\providecommand{\rbrak}[1]{\ensuremath{\left.#1\right)}}
\providecommand{\cbrak}[1]{\ensuremath{\left\{#1\right\}}}
\providecommand{\lcbrak}[1]{\ensuremath{\left\{#1\right.}}
\providecommand{\rcbrak}[1]{\ensuremath{\left.#1\right\}}}
\theoremstyle{remark}
\newtheorem{rem}{Remark}
\newcommand{\sgn}{\mathop{\mathrm{sgn}}}
\providecommand{\abs}[1]{\left\vert#1\right\vert}
\providecommand{\res}[1]{\Res\displaylimits_{#1}} 
\providecommand{\norm}[1]{\left\lVert#1\right\rVert}
%\providecommand{\norm}[1]{\lVert#1\rVert}
\providecommand{\mtx}[1]{\mathbf{#1}}
\providecommand{\mean}[1]{E\left[ #1 \right]}
\providecommand{\fourier}{\overset{\mathcal{F}}{ \rightleftharpoons}}
%\providecommand{\hilbert}{\overset{\mathcal{H}}{ \rightleftharpoons}}
\providecommand{\system}{\overset{\mathcal{H}}{ \longleftrightarrow}}
	%\newcommand{\solution}[2]{\textbf{Solution:}{#1}}
\newcommand{\solution}{\noindent \textbf{Solution: }}
\newcommand{\cosec}{\,\text{cosec}\,}
\providecommand{\dec}[2]{\ensuremath{\overset{#1}{\underset{#2}{\gtrless}}}}
\newcommand{\myvec}[1]{\ensuremath{\begin{pmatrix}#1\end{pmatrix}}}
\newcommand{\mydet}[1]{\ensuremath{\begin{vmatrix}#1\end{vmatrix}}}
\numberwithin{equation}{section}
\numberwithin{figure}{section}
\numberwithin{table}{section}
%\numberwithin{equation}{subsection}
%\numberwithin{problem}{section}
%\numberwithin{definition}{section}
\makeatletter
\@addtoreset{figure}{problem}
\makeatother
\let\StandardTheFigure\thefigure
\let\vec\mathbf
%\renewcommand{\thefigure}{\theproblem.\arabic{figure}}
\renewcommand{\thefigure}{\theproblem}
%\setlist[enumerate,1]{before=\renewcommand\theequation{\theenumi.\arabic{equation}}
%\counterwithin{equation}{enumi}
%\renewcommand{\theequation}{\arabic{subsection}.\arabic{equation}}
\def\putbox#1#2#3{\makebox[0in][l]{\makebox[#1][l]{}\raisebox{\baselineskip}[0in][0in]{\raisebox{#2}[0in][0in]{#3}}}}
     \def\rightbox#1{\makebox[0in][r]{#1}}
     \def\centbox#1{\makebox[0in]{#1}}
     \def\topbox#1{\raisebox{-\baselineskip}[0in][0in]{#1}}
     \def\midbox#1{\raisebox{-0.5\baselineskip}[0in][0in]{#1}}
\vspace{3cm}
	\section{NCERT-Class-9-Statistics}
\textbf{\underline{Example 9:}} In a city, the weekly observations made in a study on the cost of living index are given in the following table:
\begin{table}[!htb]
\input{tables/table_1.tex}
\end{table}\\
Draw the frequency polygon for the data above (without constructing a histogram).\\
	\textbf{\underline{Solution:} }
	\itemize
	\item\textbf{\underline{Constructing a Histogram}} We can come up with a histogram for a better understanding on the arranged data.\\ Representing the data, as per the given data.
	\begin{figure}[!ht]
		\centering
		\includegraphics[width=\columnwidth]{Fig_1}
		\label{fig:1}
		\caption{Histogram from the given data above.}
	\end{figure}
	\item\textbf{\underline{Constructing a Frequency Polygon}} Finding the class-marks of the above classes.\\
	\itemize
	\item For 140-150, the upper-limit is 150 and the lower limit is 140.
	\begin{align*}
	\text{So the class-mark}=\frac{150+140}{2}=\frac{290}{2}=145.
	\end{align*}
	\item For 150-160, the upper-limit is 160 and the lower limit is 150.
	\begin{align*}
	\text{So the class-mark}=\frac{160+150}{2}=\frac{310}{2}=155.
	\end{align*}
	\item For 160-170, the upper-limit is 170 and the lower limit is 160.
	\begin{align*}
	\text{So the class-mark}=\frac{170+160}{2}=\frac{330}{2}=165.
	\end{align*}
	\item For 170-180, the upper-limit is 180 and the lower limit is 170.
	\begin{align*}
	\text{So the class-mark}=\frac{180+170}{2}=\frac{350}{2}=175.
	\end{align*}
	\item For 180-190, the upper-limit is 190 and the lower limit is 180.
	\begin{align*}
	\text{So the class-mark}=\frac{190+180}{2}=\frac{370}{2}=185.
	\end{align*}
	\item For 190-200, the upper-limit is 200 and the lower limit is 190.
	\begin{align*}
	\text{So the class-mark}=\frac{200+190}{2}=\frac{390}{2}=195.
	\end{align*}
	Tabulating the above class-marks corresponding to their classes.
	\begin{table}[!htb]
\input{tables/table_2.tex}
\end{table}\\
We can now draw a frequency polygon by plotting the class-marks along the horizontal axis, the frequency along the vertical axis, and then plotting and joining the points B(145,5), C(155,10), D(165,20), E(175,9), F(185,6) and G(195,2) by line segments.\\
\\
\fbox{\begin{minipage}{18.5em}
\textbf{Note:} We should plot the point corresponding to the class-mark of the class 130-140 (just before the lowest class 140-150) with zero frequency, i.e., A(135,0), and the point H(205,0) occurs immediately after G(195,2).
\end{minipage}}\\
\\
Therefore, plotting the resulting frequency polygon as ABCDEFGH. The shaded portion represents the polygon required.
\begin{figure}[htb] 
		\centering
		\includegraphics[width=\columnwidth]{Fig_2}
		\caption{Frequency polygon from the given data above.}
	\end{figure}
\end{document}