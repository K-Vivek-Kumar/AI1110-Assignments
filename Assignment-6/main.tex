\let\negmedspace\undefined
\let\negthickspace\undefined
%\RequirePackage{amsmath}
\documentclass[journal,12pt,twocolumn]{IEEEtran}
%
% \usepackage{setspace}
 \usepackage{gensymb}
%\doublespacing
 \usepackage{polynom}
%\singlespacing
%\usepackage{silence}
%Disable all warnings issued by latex starting with "You have..."
%\usepackage{graphicx}
\usepackage{amssymb}
%\usepackage{relsize}
\usepackage[cmex10]{amsmath}
%\usepackage{amsthm}
%\interdisplaylinepenalty=2500
%\savesymbol{iint}
%\usepackage{txfonts}
%\restoresymbol{TXF}{iint}
%\usepackage{wasysym}
\usepackage{amsthm}
%\usepackage{pifont}
%\usepackage{iithtlc}
% \usepackage{mathrsfs}
% \usepackage{txfonts}
 \usepackage{stfloats}
% \usepackage{steinmetz}
 \usepackage{bm}
% \usepackage{cite}
% \usepackage{cases}
% \usepackage{subfig}
%\usepackage{xtab}
\usepackage{longtable}
%\usepackage{multirow}
%\usepackage{algorithm}
%\usepackage{algpseudocode}
\usepackage{enumitem}
 \usepackage{mathtools}
 \usepackage{tikz}
% \usepackage{circuitikz}
% \usepackage{verbatim}
%\usepackage{tfrupee}
\usepackage[breaklinks=true]{hyperref}
%\usepackage{stmaryrd}
%\usepackage{tkz-euclide} % loads  TikZ and tkz-base
%\usetkzobj{all}
\usepackage{listings}
    \usepackage{color}                                            %%
    \usepackage{array}                                            %%
    \usepackage{longtable}                                        %%
    \usepackage{calc}                                             %%
    \usepackage{multirow}                                         %%
    \usepackage{hhline}                                           %%
    \usepackage{ifthen}   
    \usepackage[english]{babel}
    \usepackage{graphicx}                                        %%
  %optionally (for landscape tables embedded in another document): %%
    \usepackage{lscape}
% \usepackage{multicol}
% \usepackage{chngcntr}
%\usepackage{enumerate}

%\usepackage{wasysym}
%\newcounter{MYtempeqncnt}
\DeclareMathOperator*{\Res}{Res}
\DeclareMathOperator*{\equals}{=}
%\renewcommand{\baselinestretch}{2}

%\renewcommand\thesection{\arabic{section}}
%\renewcommand\thesubsection{\thesection.\arabic{subsection}}
%\renewcommand\thesubsubsection{\thesubsection.\arabic{subsubsection}}

%\renewcommand\thesectiondis{\arabic{section}}
%\renewcommand\thesubsectiondis{\thesectiondis.\arabic{subsection}}
%\renewcommand\thesubsubsectiondis{\thesubsectiondis.\arabic{subsubsection}}

% correct bad hyphenation here
\hyphenation{op-tical net-works semi-conduc-tor}
\def\inputGnumericTable{}                                 %%

\lstset{
%language=C,
frame=single, 
breaklines=true,
columns=fullflexible
}
%\lstset{
%language=tex,
%frame=single, 
%breaklines=true
%}
\title{Assignment 6}
\author{K Vivek Kumar - CS21BTECH11026}

\begin{document}
\date{May 12,2022}
\maketitle
%
\newtheorem{theorem}{Theorem}[section]
\newtheorem{problem}{Problem}
\newtheorem{proposition}{Proposition}[section]
\newtheorem{lemma}{Lemma}[section]
\newtheorem{corollary}[theorem]{Corollary}
\newtheorem{example}{Example}[section]
\newtheorem{definition}[problem]{Definition}
%\newtheorem{thm}{Theorem}[section] 
%\newtheorem{defn}[thm]{Definition}
%\newtheorem{algorithm}{Algorithm}[section]
%\newtheorem{cor}{Corollary}
\newcommand{\BEQA}{\begin{eqnarray}}
\newcommand{\EEQA}{\end{eqnarray}}
\newcommand{\define}{\stackrel{\triangle}{=}}
\newcommand*\circled[1]{\tikz[baseline=(char.base)]{
    \node[shape=circle,draw,inner sep=2pt] (char) {#1};}}
\bibliographystyle{IEEEtran}
%\bibliographystyle{ieeetr}
\providecommand{\mbf}{\mathbf}
\providecommand{\pr}[1]{\ensuremath{\Pr\left(#1\right)}}
\providecommand{\qfunc}[1]{\ensuremath{Q\left(#1\right)}}
\providecommand{\sbrak}[1]{\ensuremath{{}\left[#1\right]}}
\providecommand{\lsbrak}[1]{\ensuremath{{}\left[#1\right.}}
\providecommand{\rsbrak}[1]{\ensuremath{{}\left.#1\right]}}
\providecommand{\brak}[1]{\ensuremath{\left(#1\right)}}
\providecommand{\lbrak}[1]{\ensuremath{\left(#1\right.}}
\providecommand{\rbrak}[1]{\ensuremath{\left.#1\right)}}
\providecommand{\cbrak}[1]{\ensuremath{\left\{#1\right\}}}
\providecommand{\lcbrak}[1]{\ensuremath{\left\{#1\right.}}
\providecommand{\rcbrak}[1]{\ensuremath{\left.#1\right\}}}
\theoremstyle{remark}
\newtheorem{rem}{Remark}
\newcommand{\sgn}{\mathop{\mathrm{sgn}}}
\providecommand{\abs}[1]{\left\vert#1\right\vert}
\providecommand{\res}[1]{\Res\displaylimits_{#1}} 
\providecommand{\norm}[1]{\left\lVert#1\right\rVert}
%\providecommand{\norm}[1]{\lVert#1\rVert}
\providecommand{\mtx}[1]{\mathbf{#1}}
\providecommand{\mean}[1]{E\left[ #1 \right]}
\providecommand{\fourier}{\overset{\mathcal{F}}{ \rightleftharpoons}}
%\providecommand{\hilbert}{\overset{\mathcal{H}}{ \rightleftharpoons}}
\providecommand{\system}{\overset{\mathcal{H}}{ \longleftrightarrow}}
	%\newcommand{\solution}[2]{\textbf{Solution:}{#1}}
\newcommand{\solution}{\noindent \textbf{Solution: }}
\newcommand{\cosec}{\,\text{cosec}\,}
\providecommand{\dec}[2]{\ensuremath{\overset{#1}{\underset{#2}{\gtrless}}}}
\newcommand{\myvec}[1]{\ensuremath{\begin{pmatrix}#1\end{pmatrix}}}
\newcommand{\mydet}[1]{\ensuremath{\begin{vmatrix}#1\end{vmatrix}}}
\numberwithin{equation}{section}
\numberwithin{figure}{section}
\numberwithin{table}{section}
%\numberwithin{equation}{subsection}
%\numberwithin{problem}{section}
%\numberwithin{definition}{section}
\makeatletter
\@addtoreset{figure}{problem}
\makeatother
\let\StandardTheFigure\thefigure
\let\vec\mathbf
%\renewcommand{\thefigure}{\theproblem.\arabic{figure}}
\renewcommand{\thefigure}{\theproblem}
%\setlist[enumerate,1]{before=\renewcommand\theequation{\theenumi.\arabic{equation}}
%\counterwithin{equation}{enumi}
%\renewcommand{\theequation}{\arabic{subsection}.\arabic{equation}}
\def\putbox#1#2#3{\makebox[0in][l]{\makebox[#1][l]{}\raisebox{\baselineskip}[0in][0in]{\raisebox{#2}[0in][0in]{#3}}}}
     \def\rightbox#1{\makebox[0in][r]{#1}}
     \def\centbox#1{\makebox[0in]{#1}}
     \def\topbox#1{\raisebox{-\baselineskip}[0in][0in]{#1}}
     \def\midbox#1{\raisebox{-0.5\baselineskip}[0in][0in]{#1}}
\vspace{3cm}
	\section{Class-12-Probability-Exercise-13.4} \par\textbf{\underline{Question 1:}} State which of the following are not the probability distributions of a random variable. Give reasons for your answer.
	\begin{enumerate}[label=(\roman*)]
    \item Random variable: X
    \begin{table}[!h] 
	      \input{tables/prob_1.tex}
	      \end{table}
	      \item Random variable: X
    \begin{table}[!h] 
	      \input{tables/prob_2.tex}
	      \end{table}
	      \item Random variable: Y
    \begin{table}[!h] 
	      \input{tables/prob_3.tex}
	      \end{table}
	      \item Random variable: Z
    \begin{table}[!h] 
	      \input{tables/prob_4.tex}
	      \end{table}
	\end{enumerate}
	\par\textbf{\underline{Solution:} }We can verify whether a probability distribution is valid for a given random variable by checking two of its properties. The below are the one to be verified in each case for a random variable X.\\
	\\
	\fbox{\begin{minipage}{20.5em}
\textbf{Property 1:} The value of P(X) should always be positive.
\begin{align*}
p_i>0, \text{for $i = \cbrak{1,2,3,...,n}$}
\end{align*}
\end{minipage}}
\\
\fbox{\begin{minipage}{20.5em}
\textbf{Property 2:} The sum of all the values of P(X) should always sum upto one.
\begin{align*}
\sum^{n}_{i = 1}p_i = 1, \text{for $i = \cbrak{1,2,3,...,n}$}
\end{align*}
\end{minipage}}\\
\begin{enumerate}[label=(\roman*)]
    \item For the random variable X, we can observe that all the $p_i$ are positive, and also
    \begin{figure}[htb] 
		\centering
		\includegraphics[width=\columnwidth]{Figure_1}
	\end{figure}
    \begin{align}
    p_1 + p_2 + p_3 &= 0.4 + 0.4 + 0.2\\ 
   p_1 + p_2 + p_3 &= 1
    \end{align}
    And also from the PMF graph we plotted, its clear that all values are positive and the CDF also approaches to 1.\\
    $\therefore$ This probability distribution of the random variable X is a valid one.
	      \item For this probability distribution we can observe that the value of $p_4$ i.e., -0.1 is a negative value, which violates the first property of a probability distribution.\\
	      \begin{figure}[htb] 
		\centering
		\includegraphics[width=\columnwidth]{Figure_2}
	\end{figure}
	And also from the PMF graph we plotted, its clear that the value of the probability at the x-axis point 3 is negative.\\
	      $\therefore$ This probability distribution of the random variable X is \textbf{NOT} a valid one.\\
	      \item For the random variable Y, we can observe that all the $p_i$ are positive, and also
	      \begin{align}
	      p_1 + p_2 + p_3 &= 0.6 + 0.1 + 0.2\\
	       p_1 + p_2 + p_3 &= 0.9 < 1
	      \end{align}
	      \begin{figure}[htb] 
		\centering
		\includegraphics[width=\columnwidth]{Figure_3}
	\end{figure}
	      Though the property 1 is valid here, but the property 2 isn't a valid one. The sum is not coming out to be 1.\\
	      And also from the PMF graph we plotted, its clear that all values are positive and the CDF doesn't approach the value 1.\\
	      $\therefore$ This probability distribution of the random variable Y is \textbf{NOT} a valid one.\\
	      \item For the random variable Z, we can observe that all the $p_i$ are positive, and also
 \begin{align}
 \begin{split}
p_1 + p_2 + p_3 + p_4 + p_5 ={}& 0.3 + 0.2 + 0.4 \\
                               &+ 0.1 + 0.05
 \end{split}\\
p_1 + p_2 + p_3 + p_4 + p_5 ={}& 1.05 > 1
 \end{align}
 \begin{figure}[htb] 
		\centering
		\includegraphics[width=\columnwidth]{Figure_4}
	\end{figure}
	      Though the property 1 is valid here, but the property 2 isn't a valid one. The sum is not coming out to be 1.\\
	      And also from the PMF graph we plotted, its clear that all values are positive and the CDF exceeds the value 1.\\
	      $\therefore$ This probability distribution of the random variable Z is \textbf{NOT} a valid one.\\
	\end{enumerate}
\end{document}