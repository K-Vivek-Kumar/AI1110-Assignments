\let\negmedspace\undefined
\let\negthickspace\undefined
%\RequirePackage{amsmath}
\documentclass[journal,12pt,twocolumn]{IEEEtran}
%
% \usepackage{setspace}
 \usepackage{gensymb}
%\doublespacing
 \usepackage{polynom}
%\singlespacing
%\usepackage{silence}
%Disable all warnings issued by latex starting with "You have..."
%\usepackage{graphicx}
\usepackage{amssymb}
%\usepackage{relsize}
\usepackage[cmex10]{amsmath}
%\usepackage{amsthm}
%\interdisplaylinepenalty=2500
%\savesymbol{iint}
%\usepackage{txfonts}
%\restoresymbol{TXF}{iint}
\usepackage{amsmath}
%\usepackage{wasysym}
\usepackage{amsthm}
%\usepackage{pifont}
%\usepackage{iithtlc}
% \usepackage{mathrsfs}
% \usepackage{txfonts}
 \usepackage{stfloats}
% \usepackage{steinmetz}
 \usepackage{bm}
% \usepackage{cite}
% \usepackage{cases}
% \usepackage{subfig}
%\usepackage{xtab}
\usepackage{longtable}
%\usepackage{multirow}
%\usepackage{algorithm}
%\usepackage{algpseudocode}
\usepackage{enumitem}
 \usepackage{mathtools}
 \usepackage{tikz}
% \usepackage{circuitikz}
% \usepackage{verbatim}
%\usepackage{tfrupee}
\usepackage[breaklinks=true]{hyperref}
%\usepackage{stmaryrd}
%\usepackage{tkz-euclide} % loads  TikZ and tkz-base
%\usetkzobj{all}
\usepackage{listings}
    \usepackage{color}                                            %%
    \usepackage{array}                                            %%
    \usepackage{longtable}                                        %%
    \usepackage{calc}                                             %%
    \usepackage{multirow}                                         %%
    \usepackage{hhline}                                           %%
    \usepackage{ifthen}   
    \usepackage[english]{babel}
    \usepackage{graphicx}                                        %%
  %optionally (for landscape tables embedded in another document): %%
    \usepackage{lscape}
% \usepackage{multicol}
% \usepackage{chngcntr}
%\usepackage{enumerate}

%\usepackage{wasysym}
%\newcounter{MYtempeqncnt}
\DeclareMathOperator*{\Res}{Res}
\DeclareMathOperator*{\equals}{=}
%\renewcommand{\baselinestretch}{2}

%\renewcommand\thesection{\arabic{section}}
%\renewcommand\thesubsection{\thesection.\arabic{subsection}}
%\renewcommand\thesubsubsection{\thesubsection.\arabic{subsubsection}}

%\renewcommand\thesectiondis{\arabic{section}}
%\renewcommand\thesubsectiondis{\thesectiondis.\arabic{subsection}}
%\renewcommand\thesubsubsectiondis{\thesubsectiondis.\arabic{subsubsection}}

% correct bad hyphenation here
\hyphenation{op-tical net-works semi-conduc-tor}
\def\inputGnumericTable{}                                 %%

\lstset{
%language=C,
frame=single, 
breaklines=true,
columns=fullflexible
}
%\lstset{
%language=tex,
%frame=single, 
%breaklines=true
%}
\title{Assignment 5}
\author{K Vivek Kumar - CS21BTECH11026}

\begin{document}
\date{May 9,2022}
\maketitle
%
\newtheorem{theorem}{Theorem}[section]
\newtheorem{problem}{Problem}
\newtheorem{proposition}{Proposition}[section]
\newtheorem{lemma}{Lemma}[section]
\newtheorem{corollary}[theorem]{Corollary}
\newtheorem{example}{Example}[section]
\newtheorem{definition}[problem]{Definition}
%\newtheorem{thm}{Theorem}[section] 
%\newtheorem{defn}[thm]{Definition}
%\newtheorem{algorithm}{Algorithm}[section]
%\newtheorem{cor}{Corollary}
\newcommand{\BEQA}{\begin{eqnarray}}
\newcommand{\EEQA}{\end{eqnarray}}
\newcommand{\define}{\stackrel{\triangle}{=}}
\newcommand*\circled[1]{\tikz[baseline=(char.base)]{
    \node[shape=circle,draw,inner sep=2pt] (char) {#1};}}
\bibliographystyle{IEEEtran}
%\bibliographystyle{ieeetr}
\providecommand{\mbf}{\mathbf}
\providecommand{\pr}[1]{\ensuremath{\Pr\left(#1\right)}}
\providecommand{\qfunc}[1]{\ensuremath{Q\left(#1\right)}}
\providecommand{\sbrak}[1]{\ensuremath{{}\left[#1\right]}}
\providecommand{\lsbrak}[1]{\ensuremath{{}\left[#1\right.}}
\providecommand{\rsbrak}[1]{\ensuremath{{}\left.#1\right]}}
\providecommand{\brak}[1]{\ensuremath{\left(#1\right)}}
\providecommand{\lbrak}[1]{\ensuremath{\left(#1\right.}}
\providecommand{\rbrak}[1]{\ensuremath{\left.#1\right)}}
\providecommand{\cbrak}[1]{\ensuremath{\left\{#1\right\}}}
\providecommand{\lcbrak}[1]{\ensuremath{\left\{#1\right.}}
\providecommand{\rcbrak}[1]{\ensuremath{\left.#1\right\}}}
\theoremstyle{remark}
\newtheorem{rem}{Remark}
\newcommand{\sgn}{\mathop{\mathrm{sgn}}}
\providecommand{\abs}[1]{\left\vert#1\right\vert}
\providecommand{\res}[1]{\Res\displaylimits_{#1}} 
\providecommand{\norm}[1]{\left\lVert#1\right\rVert}
%\providecommand{\norm}[1]{\lVert#1\rVert}
\providecommand{\mtx}[1]{\mathbf{#1}}
\providecommand{\mean}[1]{E\left[ #1 \right]}
\providecommand{\fourier}{\overset{\mathcal{F}}{ \rightleftharpoons}}
%\providecommand{\hilbert}{\overset{\mathcal{H}}{ \rightleftharpoons}}
\providecommand{\system}{\overset{\mathcal{H}}{ \longleftrightarrow}}
	%\newcommand{\solution}[2]{\textbf{Solution:}{#1}}
\newcommand{\solution}{\noindent \textbf{Solution: }}
\newcommand{\cosec}{\,\text{cosec}\,}
\providecommand{\dec}[2]{\ensuremath{\overset{#1}{\underset{#2}{\gtrless}}}}
\newcommand{\myvec}[1]{\ensuremath{\begin{pmatrix}#1\end{pmatrix}}}
\newcommand{\mydet}[1]{\ensuremath{\begin{vmatrix}#1\end{vmatrix}}}
\numberwithin{equation}{section}
\numberwithin{figure}{section}
\numberwithin{table}{section}
%\numberwithin{equation}{subsection}
\newcommand*{\Perm}[2]{{}^{#1}\!P_{#2}}%
\newcommand*{\Comb}[2]{{}^{#1}C_{#2}}%

%\newcommand\Myperm[2][^n]{\prescript{#1\mkern-2.5mu}{}P_{#2}}
%\newcommand\Mycomb[2][^n]{\prescript{#1\mkern-0.5mu}{}C_{#2}}

%\numberwithin{problem}{section}
%\numberwithin{definition}{section}
\makeatletter
\@addtoreset{figure}{problem}
\makeatother
\let\StandardTheFigure\thefigure
\let\vec\mathbf
%\renewcommand{\thefigure}{\theproblem.\arabic{figure}}
\renewcommand{\thefigure}{\theproblem}
%\setlist[enumerate,1]{before=\renewcommand\theequation{\theenumi.\arabic{equation}}
%\counterwithin{equation}{enumi}
%\renewcommand{\theequation}{\arabic{subsection}.\arabic{equation}}
\def\putbox#1#2#3{\makebox[0in][l]{\makebox[#1][l]{}\raisebox{\baselineskip}[0in][0in]{\raisebox{#2}[0in][0in]{#3}}}}
     \def\rightbox#1{\makebox[0in][r]{#1}}
     \def\centbox#1{\makebox[0in]{#1}}
     \def\topbox#1{\raisebox{-\baselineskip}[0in][0in]{#1}}
     \def\midbox#1{\raisebox{-0.5\baselineskip}[0in][0in]{#1}}
\vspace{3cm}
	\section{Class-11-Probability-Miscellaneous Exercise-16}
\par\textbf{\underline{Question 5:}} Out of 100 students, two sections of 40 and 60 are formed. If you and your friend are among the 100 students, what is the probability that
    \begin{enumerate}[label=(\roman*)]
    \item you both enter the same section?
    \item you both enter different sections?
	\end{enumerate}
	\par\textbf{\underline{Solution:}} Let the students be divided into two sections, namely 'A' and 'B'. We can individually find the probability of each student using random variables.\\
	\\
	\fbox{\begin{minipage}{20.5em}
	\par \textbf{Constructing a mapping pattern} of the given data (defining pattern),
	\begin{align*}
    \text{\brak{\textbf{My class},\textbf{My friend's class}}}&\rightarrow \textbf{Real number}
    \end{align*}
	\end{minipage}}\\
	\par The total possible cases of me and my friend falling in the various sections can be given accordingly and are mapped in the following way:
	\begin{align*}
    \text{(A,A)}&\rightarrow 0  &  \text{(A,B)}&\rightarrow 1\\
    \text{(B,A)}&\rightarrow 2  &  \text{(B,B)}&\rightarrow 3
    \end{align*}
    Let's take a random variable $'X'$, such that it represents above mapped real numbers.
    \begin{align*}
    X \in \cbrak{0,1,2,3}
    \end{align*}
    \begin{enumerate}[label=(\roman*)]
    \item\textbf{Case of both of us having the same section:} The possible values of this case are (A,A) and (B,B).
    \begin{align}
\text{P}\brak{X=\cbrak{0,3}} &= \text{P}\brak{X=0} + \text{P}\brak{X=3}\\
                             &= \frac{\text{n}\brak{X = 0}}{\text{n}\brak{X}}+\frac{\text{n}\brak{X = 3}}{\text{n}\brak{X}}
\end{align}
Getting the same section A (say of 40 students), can be expressed as a form of choosing two students out of the 40 and allotting them the section 'A'. Similarly in the case of section B, we would be choosing two out of the 60 students to allot them that section.\\
\par Therefore, the number of cases in which we would be getting section A is $\Comb{40}{2}$. And also the number of case in which we would be getting the section B is $\Comb{60}{2}$.\\
\begin{align}
\text{P}\brak{X=\cbrak{0,3}} &= \frac{\Comb{40}{2}}{\text{n}\brak{X}} + \frac{\Comb{60}{2}}{\text{n}\brak{X}}
\end{align}
And as we belong to the 100 students, so the total number of cases of selecting a section for us can be given by $\Comb{100}{2}$.
\begin{align}
\text{P}\brak{X=\cbrak{0,3}} &= \frac{\Comb{40}{2}}{\Comb{100}{2}} + \frac{\Comb{60}{2}}{\Comb{100}{2}}  \quad \\
                             &= \frac{39 \times 40}{99 \times 100}+\frac{59 \times 60}{99 \times 100}\\
\therefore\text{P}\brak{X=\cbrak{0,3}} &= \frac{17}{33}
    \end{align}
    \item\textbf{Case of both of us having different sections:} The possible values of this case are (A,B) and (B,A).
    \begin{align*}
    \because X \in \cbrak{0,1,2,3}
    \end{align*}
    This can also be written by excluding the cases of the previous sub-problem,
    \begin{align}
    \text{P}\brak{X=\cbrak{1,2}} &= 1-\text{P}\brak{X=\cbrak{0,3}}\\
                                 &= 1-\frac{17}{33}\\
\therefore\text{P}\brak{X=\cbrak{1,2}} &= \frac{16}{33}
    \end{align}
	\end{enumerate}
	\par Therefore, the probability of me and my friend to be in the same section is equal to $\frac{17}{33}$, whereas, the probability of me and my friend being in different sections is $\frac{16}{33}$.
\end{document}