%%%%%%%%%%%%%%%%%%%%%%%%%%%%%%%%%%%%%%%%%%%%%%%%%%%%%%%%%%%%%%%
%
% Welcome to Overleaf --- just edit your LaTeX on the left,
% and we'll compile it for you on the right. If you open the
% 'Share' menu, you can invite other users to edit at the same
% time. See www.overleaf.com/learn for more info. Enjoy!
%
%%%%%%%%%%%%%%%%%%%%%%%%%%%%%%%%%%%%%%%%%%%%%%%%%%%%%%%%%%%%%%%





% Inbuilt themes in beamer
\documentclass{beamer}
% Theme choice:
\usetheme{CambridgeUS}
% Title page details: 
\title{Assignment-6} 
\author{K Vivek Kumar - CS21BTECH11026}
\date{\today}
\logo{\large \LaTeX{}}


\begin{document}

% Title page frame
\begin{frame}
    \titlepage 
\end{frame}

% Remove logo from the next slides
\logo{}


% Outline frame
\begin{frame}{Papoulis-Exercise-5}
TABLE of CONTENTS
    \tableofcontents
\end{frame}


% Lists frame
\section{Question}
\begin{frame}{Problem 5-7}
We place at random 200 points in the interval (0, 100). The distance from 0 to the first random point is a random variable $z$. Find $F_{z}(z)$.
	\begin{enumerate}
	\item exactly, and 
	\item using the Poisson approximation. 
	\end{enumerate}
\end{frame}


% Blocks frame
\section{Solution : Theorems involved - 1}
\begin{frame}{Solution}
The following theorem would be involved in the problem.\\
\begin{block}{FUNDAMENTAL THEOREM:}
\begin{align}
p_{n}(k)&=\text{P($A$ occurs $k$ times in any order)} \\
        &= \begin{pmatrix}
n\\
k
\end{pmatrix}p^{k}q^{n-k}
\end{align}
    \end{block}
\end{frame} 
\section{Solution : Theorems involved - 2}
\begin{frame}{Solution}
The following theorem would be involved in the problem.\\
\begin{block}{POISSON THEOREM:} If 
\begin{align*}
n \to \infty; && p \to 0; && \text{such that } np \to \lambda
\end{align*}
then for $k=0,1,2..$,
\begin{align}
\frac{n!}{k!(n-k)!}p^{k}q^{n-k}\xrightarrow[\text{$n\to \infty$}]{} e^{-\lambda}\frac{\lambda^{k}}{k!}
\end{align}
    \end{block}
\end{frame}
\section{Solution : deductions}
\begin{frame}{Solution : deductions}
Clearly, $z' \leq z$ iff the number n$(0,z)$ of the points in the interval $(0,z)$ is atleast one.\\
	Hence,
	\begin{align}
	F_{z}(z) &= \text{P}(z' \leq z) \\
	         &= \text{P}(\text{n'}(0,z)>0)\\
	F_{z}(z) &= 1-\text{P}(\text{n'}(0,z)=0)
	\end{align}
	The probability p that a particular point is in the interval $(0,z)$ equals $\frac{z}{100}$. With $n=200,k=0$ and $p=\frac{z}{100}$, from the fundamental theorem it gives $\text{P}(\text{n'}(0,z)=0)=(1-p)^{200}$.\\
\end{frame} 
\section{Solution (i)}
\begin{frame}{Solution (i)} Hence from the fundamental theorem, we can conclude that
\begin{align}
	F_{z}(z) &= 1-(1-\frac{z}{100})^{200}
	\end{align}

\end{frame}
\section{Solution (ii)}
\begin{frame}{Solution (ii)}
From the poisson's theorem, it follows that for $z \ll 100$,
	\begin{align}
	F_{z}(z) \simeq 1-e^{-2z}
\end{align}
\end{frame}

\end{document}