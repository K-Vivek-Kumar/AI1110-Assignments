\let\negmedspace\undefined
\let\negthickspace\undefined
%\RequirePackage{amsmath}
\documentclass[journal,12pt,twocolumn]{IEEEtran}
%
% \usepackage{setspace}
 \usepackage{gensymb}
%\doublespacing
 \usepackage{polynom}
%\singlespacing
%\usepackage{silence}
%Disable all warnings issued by latex starting with "You have..."
%\usepackage{graphicx}
\usepackage{amssymb}
%\usepackage{relsize}
\usepackage[cmex10]{amsmath}
%\usepackage{amsthm}
%\interdisplaylinepenalty=2500
%\savesymbol{iint}
%\usepackage{txfonts}
%\restoresymbol{TXF}{iint}
%\usepackage{wasysym}
\usepackage{amsthm}
%\usepackage{pifont}
%\usepackage{iithtlc}
% \usepackage{mathrsfs}
% \usepackage{txfonts}
 \usepackage{stfloats}
% \usepackage{steinmetz}
 \usepackage{bm}
% \usepackage{cite}
% \usepackage{cases}
% \usepackage{subfig}
%\usepackage{xtab}
\usepackage{longtable}
%\usepackage{multirow}
%\usepackage{algorithm}
%\usepackage{algpseudocode}
\usepackage{enumitem}
 \usepackage{mathtools}
 \usepackage{tikz}
% \usepackage{circuitikz}
% \usepackage{verbatim}
%\usepackage{tfrupee}
\usepackage[breaklinks=true]{hyperref}
%\usepackage{stmaryrd}
%\usepackage{tkz-euclide} % loads  TikZ and tkz-base
%\usetkzobj{all}
\usepackage{listings}
    \usepackage{color}                                            %%
    \usepackage{array}                                            %%
    \usepackage{longtable}                                        %%
    \usepackage{calc}                                             %%
    \usepackage{multirow}                                         %%
    \usepackage{hhline}                                           %%
    \usepackage{ifthen}   
    \usepackage[english]{babel}
    \usepackage{graphicx}                                        %%
  %optionally (for landscape tables embedded in another document): %%
    \usepackage{lscape}
% \usepackage{multicol}
% \usepackage{chngcntr}
%\usepackage{enumerate}

%\usepackage{wasysym}
%\newcounter{MYtempeqncnt}
\DeclareMathOperator*{\Res}{Res}
\DeclareMathOperator*{\equals}{=}
%\renewcommand{\baselinestretch}{2}

%\renewcommand\thesection{\arabic{section}}
%\renewcommand\thesubsection{\thesection.\arabic{subsection}}
%\renewcommand\thesubsubsection{\thesubsection.\arabic{subsubsection}}

%\renewcommand\thesectiondis{\arabic{section}}
%\renewcommand\thesubsectiondis{\thesectiondis.\arabic{subsection}}
%\renewcommand\thesubsubsectiondis{\thesubsectiondis.\arabic{subsubsection}}

% correct bad hyphenation here
\hyphenation{op-tical net-works semi-conduc-tor}
\def\inputGnumericTable{}                                 %%

\lstset{
%language=C,
frame=single, 
breaklines=true,
columns=fullflexible
}
%\lstset{
%language=tex,
%frame=single, 
%breaklines=true
%}
\title{Assignment 4}
\author{K Vivek Kumar - CS21BTECH11026}

\begin{document}
\date{April 29,2022}
\maketitle
%
\newtheorem{theorem}{Theorem}[section]
\newtheorem{problem}{Problem}
\newtheorem{proposition}{Proposition}[section]
\newtheorem{lemma}{Lemma}[section]
\newtheorem{corollary}[theorem]{Corollary}
\newtheorem{example}{Example}[section]
\newtheorem{definition}[problem]{Definition}
%\newtheorem{thm}{Theorem}[section] 
%\newtheorem{defn}[thm]{Definition}
%\newtheorem{algorithm}{Algorithm}[section]
%\newtheorem{cor}{Corollary}
\newcommand{\BEQA}{\begin{eqnarray}}
\newcommand{\EEQA}{\end{eqnarray}}
\newcommand{\define}{\stackrel{\triangle}{=}}
\newcommand*\circled[1]{\tikz[baseline=(char.base)]{
    \node[shape=circle,draw,inner sep=2pt] (char) {#1};}}
\bibliographystyle{IEEEtran}
%\bibliographystyle{ieeetr}
\providecommand{\mbf}{\mathbf}
\providecommand{\pr}[1]{\ensuremath{\Pr\left(#1\right)}}
\providecommand{\qfunc}[1]{\ensuremath{Q\left(#1\right)}}
\providecommand{\sbrak}[1]{\ensuremath{{}\left[#1\right]}}
\providecommand{\lsbrak}[1]{\ensuremath{{}\left[#1\right.}}
\providecommand{\rsbrak}[1]{\ensuremath{{}\left.#1\right]}}
\providecommand{\brak}[1]{\ensuremath{\left(#1\right)}}
\providecommand{\lbrak}[1]{\ensuremath{\left(#1\right.}}
\providecommand{\rbrak}[1]{\ensuremath{\left.#1\right)}}
\providecommand{\cbrak}[1]{\ensuremath{\left\{#1\right\}}}
\providecommand{\lcbrak}[1]{\ensuremath{\left\{#1\right.}}
\providecommand{\rcbrak}[1]{\ensuremath{\left.#1\right\}}}
\theoremstyle{remark}
\newtheorem{rem}{Remark}
\newcommand{\sgn}{\mathop{\mathrm{sgn}}}
\providecommand{\abs}[1]{\left\vert#1\right\vert}
\providecommand{\res}[1]{\Res\displaylimits_{#1}} 
\providecommand{\norm}[1]{\left\lVert#1\right\rVert}
%\providecommand{\norm}[1]{\lVert#1\rVert}
\providecommand{\mtx}[1]{\mathbf{#1}}
\providecommand{\mean}[1]{E\left[ #1 \right]}
\providecommand{\fourier}{\overset{\mathcal{F}}{ \rightleftharpoons}}
%\providecommand{\hilbert}{\overset{\mathcal{H}}{ \rightleftharpoons}}
\providecommand{\system}{\overset{\mathcal{H}}{ \longleftrightarrow}}
	%\newcommand{\solution}[2]{\textbf{Solution:}{#1}}
\newcommand{\solution}{\noindent \textbf{Solution: }}
\newcommand{\cosec}{\,\text{cosec}\,}
\providecommand{\dec}[2]{\ensuremath{\overset{#1}{\underset{#2}{\gtrless}}}}
\newcommand{\myvec}[1]{\ensuremath{\begin{pmatrix}#1\end{pmatrix}}}
\newcommand{\mydet}[1]{\ensuremath{\begin{vmatrix}#1\end{vmatrix}}}
\numberwithin{equation}{section}
\numberwithin{figure}{section}
\numberwithin{table}{section}
%\numberwithin{equation}{subsection}
%\numberwithin{problem}{section}
%\numberwithin{definition}{section}
\makeatletter
\@addtoreset{figure}{problem}
\makeatother
\let\StandardTheFigure\thefigure
\let\vec\mathbf
%\renewcommand{\thefigure}{\theproblem.\arabic{figure}}
\renewcommand{\thefigure}{\theproblem}
%\setlist[enumerate,1]{before=\renewcommand\theequation{\theenumi.\arabic{equation}}
%\counterwithin{equation}{enumi}
%\renewcommand{\theequation}{\arabic{subsection}.\arabic{equation}}
\def\putbox#1#2#3{\makebox[0in][l]{\makebox[#1][l]{}\raisebox{\baselineskip}[0in][0in]{\raisebox{#2}[0in][0in]{#3}}}}
     \def\rightbox#1{\makebox[0in][r]{#1}}
     \def\centbox#1{\makebox[0in]{#1}}
     \def\topbox#1{\raisebox{-\baselineskip}[0in][0in]{#1}}
     \def\midbox#1{\raisebox{-0.5\baselineskip}[0in][0in]{#1}}
\vspace{3cm}
	\section{Class-10-Probability-Exercise-15.1}
\textbf{\underline{Question 25:}} Which of the following arguments are correct and which are not correct? Give reasons for your answer.
    \begin{enumerate}[label=(\roman*)]
    \item If coins are tossed simultaneously there are three possible outcomes-two heads, two tails or one of each. Therefore, for each of these outcomes, the probability is $\frac{1}{3}$.
    \item If a die is thrown, there are two possible outcomes-an odd number or an even number. Therefore, the probability of getting an odd number is $\frac{1}{2}$.
	\end{enumerate}
	\textbf{\underline{Solution:}} We can individually find the probability using random variables.
    \begin{enumerate}[label=(\roman*)]
    \item The possible events are (H,H); (H,T); (T,H) and (T,T).\\
    Lets take a random variable '$X$', which maps the corresponding cases.
    \begin{align*}
    \therefore X \in \cbrak{\text{(H,H),(H,T),(T,H),(T,T)}}
    \end{align*}
    For each of the following case, finding the probability.
    \begin{itemize}
    \item \textbf{Case when both are heads:}
    \begin{align}
    \text{P}\brak{X = \cbrak{\text{(H,H)}}} &= \frac{\text{n}\brak{X = \cbrak{\text{(H,H)}}}}{\text{n}\brak{X}}\\
                                              &= \frac{1}{4} = 0.25
    \end{align}
    $\therefore$ The probability of obtaining both heads is 0.25.
    \end{itemize}
    \begin{itemize}
    \item \textbf{Case when both are tails:}
    \begin{align}
    \text{P}\brak{X = \cbrak{\text{(T,T)}}} &= \frac{\text{n}\brak{X = \cbrak{\text{(T,T)}}}}{\text{n}\brak{X}}\\
                                              &= \frac{1}{4} = 0.25
    \end{align}
    $\therefore$ The probability of obtaining both tails is also 0.25.
    \end{itemize}
    \begin{itemize}
    \item \textbf{Case when one is head and other is tail:}
    \begin{align}
    \text{P}\brak{\cbrak{\text{(H,T),(T,H)}}} &= \frac{\text{n}\brak{\cbrak{\text{(H,T),(T,H)}}}}{\text{n}\brak{X}}\\
                                              &= \frac{2}{4} = \frac{1}{2} = 0.5
    \end{align}
    $\therefore$ The probability of obtaining either one on each coin is 0.5.
    \end{itemize}
    Therefore, this statement is incorrect.
    \item Lets assign a random variable '$Y$', such that it maps to the possible outcomes from a die,i.e.,
    \begin{align*}
    Y \in \cbrak{1,2,3,4,5,6}
    \end{align*}
    Now for each of the following case, we can find the probability as follows.      
    \begin{itemize}
    \item \textbf{Case when there is an odd number:} The possible outcomes are $\cbrak{1,3,5}$,
    \begin{align}
    \text{P}\brak{Y = \cbrak{1,3,5}} &= \frac{\text{n}\brak{\cbrak{1,3,5}}}{\text{n}\brak{Y}}\\
                                              &= \frac{3}{6} = \frac{1}{2} = 0.5
    \end{align}
    $\therefore$ The probability of obtaining an odd number on the die is 0.5.
    \item \textbf{Case when there is an even number:} The possible outcomes are $\cbrak{2,4,6}$,
    \begin{align}
    \text{P}\brak{Y = \cbrak{2,4,6}} &= \frac{\text{n}\brak{\cbrak{2,4,6}}}{\text{n}\brak{Y}}\\
                                              &= \frac{3}{6} = \frac{1}{2} = 0.5
    \end{align}
    $\therefore$ The probability of obtaining an even number on the die is also 0.5.
    \end{itemize}
    Therefore, this statement is correct.
	\end{enumerate}
\end{document}