%%%%%%%%%%%%%%%%%%%%%%%%%%%%%%%%%%%%%%%%%%%%%%%%%%%%%%%%%%%%%%%
%
% Welcome to Overleaf --- just edit your LaTeX on the left,
% and we'll compile it for you on the right. If you open the
% 'Share' menu, you can invite other users to edit at the same
% time. See www.overleaf.com/learn for more info. Enjoy!
%
%%%%%%%%%%%%%%%%%%%%%%%%%%%%%%%%%%%%%%%%%%%%%%%%%%%%%%%%%%%%%%%





% Inbuilt themes in beamer
\documentclass{beamer}
% Theme choice:
\usetheme{CambridgeUS}
% Title page details: 
\title{Assignment-8} 
\author{K Vivek Kumar - CS21BTECH11026}
\date{\today}
\logo{\large \LaTeX{}}


\begin{document}

% Title page frame
\begin{frame}
    \titlepage 
\end{frame}

% Remove logo from the next slides
\logo{}


% Outline frame
\begin{frame}{Papoulis-Chapter-10}
TABLE of CONTENTS
    \tableofcontents
\end{frame}


% Lists frame
\section{Question}
\begin{frame}{Problem 10-17}
Find the power spectrum $S(\omega)$ of a process $x(t)$ if $S(\omega)=0$ for $\lvert\omega\rvert>\pi$ and
	\begin{align*}
	E\{x(n+m)x(n)\}=N\delta[\,m]\,
	\end{align*}
\end{frame}


% Blocks frame
\section{Solution : Properties involved - 1}
\begin{frame}{Solution : Properties involved - 1}
The following property is involved in the problem.\\
\begin{block}{Property 1:}
As we know from $T=\dfrac{\pi}{\sigma}$,
\begin{align}
	R(mT)=E\{x(nT+mT)x(nT)\}=\begin{cases}
	I & m=0\\
	\eta^{2} & m \neq 0
	\end{cases}
	\end{align}
    \end{block}
\end{frame} 
\section{Solution : Properties involved - 2}
\begin{frame}{Solution : Properties involved - 2}
The following property is involved in the problem.\\
\begin{block}{Property 2:} \begin{align}
	R(\tau)=\sum_{n=-\infty}^{\infty}R(nT)\dfrac{\sin(\sigma(\tau-nT))}{\sigma(\tau-nT)}
	\end{align}
    \end{block}
\end{frame}
\section{Solution - I}
\begin{frame}{Solution - I}
Given,
	\begin{align}
	E\{x(n+m)x(n)\}=N\delta[\,m]\,
	\end{align}
	Therefore, from the property 2, we can deduce the following equation,
	\begin{align}
	R(\tau)&=\sum_{m=-\infty}^{\infty}R(mT)\dfrac{\sin(\sigma(\tau-mT))}{\sigma(\tau-mT)}\\
	       &=\eta^{2} + (I-\eta^{2})\dfrac{\sin \sigma\tau}{\pi\tau}
	\end{align}
	\end{frame}
\section{Solution - II}
\begin{frame}{Solution - II}
	Therefore,
	\begin{align}
	S(\omega)=2\pi\eta^{2}\delta(\omega)+2\pi(I-\eta^{2})p_{\sigma}(\omega)\label{eqn-1.6}
	\end{align}
	As it is said that $S(\omega)=0$ for $|\omega|>\pi$, therefore we can say from the equation \ref{eqn-1.6}, $\eta=0$ and $I=N$. On substituting it in the equation,
	\begin{align}
	S(\omega)&=2\pi(0)^{2}\delta(\omega)+2\pi(N-(0)^{2})p_{\sigma}(\omega)\\
	         &=2\pi N p_{\sigma}(\omega)
	\end{align}
	\par$\therefore$ The power spectrum $S(\omega)$ is $2\pi N p_{\sigma}(\omega)$.
\end{frame}

\end{document}