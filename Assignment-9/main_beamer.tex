%%%%%%%%%%%%%%%%%%%%%%%%%%%%%%%%%%%%%%%%%%%%%%%%%%%%%%%%%%%%%%%
%
% Welcome to Overleaf --- just edit your LaTeX on the left,
% and we'll compile it for you on the right. If you open the
% 'Share' menu, you can invite other users to edit at the same
% time. See www.overleaf.com/learn for more info. Enjoy!
%
%%%%%%%%%%%%%%%%%%%%%%%%%%%%%%%%%%%%%%%%%%%%%%%%%%%%%%%%%%%%%%%





% Inbuilt themes in beamer
\documentclass{beamer}
% Theme choice:
\usetheme{CambridgeUS}
% Title page details: 
\title{Assignment-9} 
\author{K Vivek Kumar - CS21BTECH11026}
\date{\today}
\logo{\large \LaTeX{}}


\begin{document}

% Title page frame
\begin{frame}
    \titlepage 
\end{frame}

% Remove logo from the next slides
\logo{}


% Outline frame
\begin{frame}{Papoulis-Chapter-15}
TABLE of CONTENTS
    \tableofcontents
\end{frame}


% Lists frame
\section{Question}
\begin{frame}{Problem 15-7}
Show that the sums $s_n = x_1 + x_2 + ... + x_n$ of independent zero mean random variables form a martingale. 
\end{frame}

\section{Solution : Property involved}
\begin{frame}{Solution : Property involved}
The following property would be involved in the problem.\\
\begin{block}{Property:}
A random sequence $x_{n}$ is called a martingale if $E\{x_n=0\}$ and 
\begin{align}
E\{x_n|x_{n-1},x_{n-2},...,x_1\}=x_{n-1}
\end{align}
    \end{block}
\end{frame}


% Blocks frame
\section{Solution : I}
\begin{frame}{Solution : I}
Given,
	\begin{align}
	s_n = x_1 + x_2 + ... + x_n
	\end{align}
	where, $x_n$ are i.i.d. random variables. We have
	\begin{align}
	s_{n+1} = s_n + x_{n+1}
	\end{align}
\end{frame} 
\section{Solution : II}
\begin{frame}{Solution : II}
	So from the property we can say that,
	\begin{align}
	E\{s_{n+1}|s_{n}\}&=E\{s_{n}+x_{n+1}|s_{n}\}\\
	&=s_{n}+E\{x_{n+1}\}\\
	&=s_n
	\end{align}
	Hence, $\{s_n\}$ represents a Martingale.
\end{frame} 

\end{document}